\documentclass{article}
\usepackage[utf8]{inputenc}

\title{Architecture for a scale IoT manufacturing cell:\\
A case study on magnetic pieces treatment classification}
\author{jafd,lacmtz, adsoft }
\date{April 2021}

\begin{document}

\maketitle

\begin{abstract}

\end{abstract}

Keywords

\section{Introduction}

\subsection{Objectives}
\subsection{Problem Statement}
Section for the introduction

\section{State of the art}
\subsection{Previous IoT architectures}
Section for the state of the art

\section{Methodology}
\subsection{General architecture}
Description and image of the general Architecture

\subsection{IoT}
\subsubsection{Data acquisition boards}
\subsubsection{Sensors}
\subsubsection{Raspberry Pi}
Subsection to describe the IoT components

\subsection{Cloud}
\subsubsection{HiveMQ (Broker Server)}
\subsubsection{Node RED}
\subsubsection{PostgreSQL}
Subsection to describe the cloud components

\subsection{Inspection}
\subsubsection{OpenCV}
\subsubsection{Cameras}
Subsection to describe the vision components

\section{Case Study}
\subsection{Experimental design and Gauge Analysis}
\subsubsection{Factorial analysis}
\subsubsection{R\&R analysis for magnetic sensors}
Subsection to describe the experimental design 

\subsection{Robot }
\subsubsection{Inverse Kinematics}
\subsubsection{Predictive Model}
Subsection to describe the robot components

\subsection{Piece detection}
\subsubsection{Magnetic Field Detection}
\subsubsection{Color Detection}
Subsection to describe the detection features

\section{Results}
Results and discussion

\section{Conclusion}
Conclusion and future work

\section{References}
\end{document}
